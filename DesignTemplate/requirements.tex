% !TEX root = SystemTemplate.tex
\chapter{User Stories, Backlog and Requirements}
\section{Overview}


\subsection{Scope}


The purpose of this document is to provide a detailed description of the end user's requirements
for this software product, as well as a development plan for satisfying each of these requirements
in the program design and implementation.  This section will also contain the stakeholder information, 
initial user stories, and requirements. 



\subsection{Purpose of the System}
The purpose of the system is to provide the user with a program that runs a specified program against "test cases" that it finds in the system.  The output of each test will be compared against the already present, correct output for each test case.  A log will be kept of the results of testing and  the associated rates of success.


\section{ Stakeholder Information}


\subsection{Customer or End User (Product Owner)}
Daniel Nix is the Product Owner of this project.  He will clarify and define the End Users' needs and requirements 
for this product, as well as establish and prioritize the product backlog.

\subsection{Management or Instructor (Scrum Master)}
Lisa Woody is the Scrum Master for the project.  She is responsible for scheduling the project meetings, as well as 
determining and assigning the tasks necessary to deliver the required product.

\subsection{Developers --Testers}
Joseph Lillo is the Technical Lead and for the project.  He will be responsible for the high level design and final
testing of the program.

\section{Business Need}
This software must simplify and automate the grading process.  The product will meet that need and enable 
the end user to not only see the immediate results of a test, but also to maintain a dated record of each test
and its detailed output.

\section{Requirements and Design Constraints}

\subsection{System  Requirements}
\begin{description}
\item [$\bullet$] The program must build and run in a Linux environment.
\item [$\bullet$] The source code for programs to be "tested" will be in C++.
\item [$\bullet$] Source code will be in the "root" directory and its subdirectories.
\item [$\bullet$] A bash shell will be used to run the program
\end{description}

\subsection{Network Requirements}
None


\subsection{Development Environment Requirements}
\begin{description}
\item [$\bullet$] The application must be written in C++
\item [$\bullet$] All work must be done in Linux
\item [$\bullet$] System calls (gcc, etc.) may be used in the application.
\item [$\bullet$] The test case input will be stored in a .tst file. The accompanying
 desired output \\ will be stored in a .ans file.
\item [$\bullet$] Testing output should be in one log file which contains: \\
\hspace{4ex} Test output results (i.e., 52 different tests will produce 52 lines in the .log file) \\
\hspace{4ex} Number passed, Number failed, Percentage of success
\end{description}

\subsection{Project  Management Methodology}
There is only one customer for this application. This customer may place constraints
on meeting times and frequency of required progress reports. Aside from customer
requests meeting times and reports will be managed by the scrum master.
For the first iteration, we need to compile and test only one program. 

\begin{itemize}
\item Trello, a free web-based project management application, will be used to keep
         track of the backlogs and sprint status.
\item All parties have access to the Sprint and Product Backlogs, via Trello.
\item This particular project will be encompassed by only one Sprint.
\item The Sprint Cycle of this project is two weeks.
\item There are no restrictions on source control.
\end{itemize}

\section{User Stories}


\subsection{User Story \#1}
As a user of the program, I would like to be able to specify a program to grade that will test the program and create a record of easy to understand output.

\subsubsection{User Story \#1 Breakdown}
This application will be targeted towards instructors needing to test submitted student programs against applicable test
cases.  The application will be run from the command line, using the name of the program to be tested as an inital parameter.
For each existing test case, the program will be run using that test case's .tst file as input.  The output will be recorded and 
compared to the accompanying answer file for that test case.  A summary of the results must accompany the recorded
output in the log file created each time the application is run.

\subsection{User Story \#2} 

As a user of the program, I want to be able to test the program against test cases located in the directory tree of that program. 


\subsubsection{User Story \#2 Breakdown}
Each program to be tested will be placed into a directory that forms the root of the directory tree related to that program.
The user will have the ability to add and remove test cases  (called case\#.tst) and their accompanying desired output file
for comparison (called case\#.ans).  The application must find all of the applicable test cases and accompanying answer 
files, and test the specified program against all the test cases found.

\subsection{User Story \#3} 
As a user of the program, I want to be able to fix the problems in the program I am testing and rerun the test without losing
the previously created log file.

\subsubsection{User Story \#3 Breakdown}
A new log file containing the tested program's outputs and summary must be created each time the application is run.
This will be an important feature, enabling the user to alter the program and visualize the effects of the alterations on
the program's output and testing summary.  Each log file will be date-stamped to achieve this result.


\section{Research or Proof of Concept Results}
None


\section{Supporting Material}
None

