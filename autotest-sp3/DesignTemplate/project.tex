% !TEX root = SystemTemplate.tex


\chapter{Project Overview}
This section provides some housekeeping type of information with regard to the 
team, project, etc. 



\section{Team Members and Roles}
\subsection{Obfuscators}
\begin{description}
\item[$\bullet$ ] Joseph Lillo - Technical Lead
\item[$\bullet$ ] Daniel Nix - Product Owner
\item[$\bullet$ ] Lisa Woody - Scrum Master
\end{description}
\subsection{Whitespace Cowboys}
The team members of the Whitespace Cowboys were Kelsey Bellew, Ryan Brown, and Ryan Feather. 

\begin{itemize}
\item Kelsey Bellew was the Scrum Master.
\item Ryan Brown was the Product Owner.
\item Ryan Feather was the Technical Lead.
\end{itemize}




\section{Project  Management Approach}
This project was managed using the Agile Software Development Method, via the Scrum framework.
Trello, a web-based project management application, was used to assign tasks and keep track of 
the product backlog.  For version control, Git was used with a Bitbucket repository giving all 
team members access to the most recent code.

The sprint length for the development of this system was two weeks, with
only one sprint being necessary for completion.  The user stories were provided by the product 
owner, Daniel Nix, who actively communicated with the future application user.  These were used
to create the product backlog and to determine the requirements and components of the system.


\section{Phase  Overview}
The program was developed through four different phases:
\begin{description}
\item[1. ] Implement a directory crawl to find all of the test cases.
\item[2. ] Run the program to be tested, using the input from the test case. Store the result in an output file.
\item[3. ] For each test case found, compare the tested program's output file to the desired output file (included with each test case).
\item[4. ] Create a time-stamped log file to retain the output files and results of the test.  This will include a record of the success and failure rate of the test.
\end{description}


\section{Terminology and Acronyms}
See Table \ref{terms}
\begin{table}[tbh]
\begin{center}
\begin{tabular}{|r|l|}
\hline
    GNU and GNU Tools & The tools provided in most GNU\textbackslash Linux environments \\ \hline
    Agile Methodology & An approach to project management and software development \url{agilemanifesto.org} \\ \hline
    C++ Template Libraries & A built-in set of classes used for storing data. \\
    \hline
\end{tabular}
\caption{Defining Some Important Terms \label{terms}}
\end{center}
\end{table}
